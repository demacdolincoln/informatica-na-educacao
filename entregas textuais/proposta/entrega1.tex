\documentclass[10pt,a4paper]{article}
\usepackage[utf8]{inputenc}
\usepackage[portuguese]{babel}

% pacotes desnecessários
%\usepackage{amsmath}
%\usepackage{amsfonts}
%\usepackage{amssymb}
%\usepackage{graphicx}
%\usepackage[left=3.00cm, right=3.00cm, top=2.00cm]{geometry}

\author{Lincoln de Macêdo\\
		Renan Tomazini\\
		Kleyton}
	
\title{Jogos educativos além do rótulo}

\begin{document}
	
	\maketitle
	
	\section*{Fundamentos}
	
	\begin{quotation}
		\textit{Ninguém escapa da educação. Em casa, na rua, na igreja ou na escola, de um modo ou de muitos todos nós envolvemos pedaços da vida com ela: para aprender, para ensinar, para aprender-e-ensinar. Para saber, para fazer, para ser ou para conviver, todos os dias misturamos a vida com a educação. Com uma ou com várias: educação? Educações.} [BRANDÃO,C.R. O que é educação]
	\end{quotation}
	
	Ainda que a imposição do sistema de ensino nos leve a crer que o processo de aprendizado ocorre quase que exclusivamente em sala de aula, o fato é que é realmente muito dificil enumerar em quais situações cotidianas não há o processo de aprendizagem e por este motivo, e ainda mais com a convergêcia entre as mídias, torna-se importante levar à sala de aula elementos que também constituem o meio em que os alunos habitam e aproveitar o seu potencial.
	
	Há muito tempo que a aula servido apenas como uma longa e exaustiva palestra ou ato de copiar o que está no quadro tornou-se símbolo de desperdício de tempo de vida, o que propomos neste trabalho é essencialmente fornecer argumentos, experiências e direçoes que auxiliem os educadores no seu trabalho quanto professores a usar ferramentas com tao talto potencial quanto jogos na sala de aula, tanto com o objetivo de atrair a atençao dos alunos quanto para aprofundar a percepçao que eles tem sobre o universo que os cerca.
	
	\section*{Objetivos:}
	Discutir o uso potencial de jogos digitais que não foram criados com finalidades educativas no processo de aprendizagem e educação de pessoas em fase escolar tendo como foco o uso desta tecnologia no ensino de ciências humanas como história, geografia e filosofia.
	
	O objetivo secundário deste trabalho é analisar uma forma de uso de jogos digitais que possam favorecer a transversalidade do ensino, considerando um cenário ótimo de cooperação entre disciplinas e a existência de aulas de computação.
	
	
	\section*{Fases de desenvolvimento:}
	Itens iniciais:
	
	\begin{enumerate}
		
		\item Revisão de literatura
		
		\item Levantamento de títulos de jogos que contenham conteúdos aproveitáveis em sala de aula.
		
		\item Levantamento de sites destinados ao ensino (seja de programação ou qualquer outra área de conhecimento humano) que tenha uma abordagem próxima ao tema deste trabalho.
		
		\item Esquematização do jogo que será usado no objetivo secundário (inicialmente RPG).
		
		\item Levantamento de casos de uso de jogos digitais em sala de aula (se possível fazer entrevistas).
		
	\end{enumerate}
	
	\section*{Fases de implementação}
	
	\begin{itemize}
		\item Criar um canal no YouTube para abordar jogos em sala de aula
		
		\item Criar um blog para divulgar o uso de jogos não "educativos" como ferramentas de apoio aos professores
		
		\item Ao menos tentar fazer um experimento numa sala de aula real
		
		\item Escrever um artigo cujo título dependerá dos items acima descritos
	\end{itemize}
	
\end{document}