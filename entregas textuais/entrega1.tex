\documentclass[10pt,a4paper]{article}
\usepackage[utf8]{inputenc}
\usepackage[portuguese]{babel}

% pacotes desnecessários
%\usepackage{amsmath}
%\usepackage{amsfonts}
%\usepackage{amssymb}
%\usepackage{graphicx}
%\usepackage[left=3.00cm, right=3.00cm, top=2.00cm]{geometry}

\author{Lincoln de Macêdo\\
		Renan Tomazini\\
		Kleyton}
	
\title{Informática na educação}

\begin{document}
	
	\maketitle
	
	\section*{Objetivos:}
	Discutir o uso potencial de jogos digitais que não foram criados com finalidades educativas no processo de aprendizagem e educação de pessoas em fase escolar tendo como foco o uso desta tecnologia no ensino de ciências humanas como história, geografia e filosofia.
	
	O objetivo secundário deste trabalho é analisar uma forma de uso de jogos digitais que possam favorecer a transversalidade do ensino, considerando um cenário ótimo de cooperação entre disciplinas e a existência de aulas de computação.
	
	\section*{Fases de desenvolvimento:}
	Itens iniciais:
	
	\begin{enumerate}
		
		\item Revisão de literatura
		
		\item Levantamento de títulos de jogos que contenham conteúdos aproveitáveis em sala de aula.
		
		\item Levantamento de sites destinados ao ensino (seja de programação ou qualquer outra área de conhecimento humano) que tenha uma abordagem próxima ao tema deste trabalho.
		
		\item Esquematização do jogo que será usado no objetivo secundário (inicialmente RPG).
		
		\item Levantamento de casos de uso de jogos digitais em sala de aula (se possível fazer entrevistas).
		
	\end{enumerate}
	
	As demais fases do trabalho dependerão dos dados coletados e conteúdos desenvolvidos nestas três fases iniciais.
	
\end{document}